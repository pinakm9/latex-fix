\documentclass[reqno]{siamart220329}
\usepackage[english]{babel}
\usepackage{lipsum}%
\makeatletter

\newcommand{\authorfootnotes}{\renewcommand\thefootnote{\@fnsymbol\c@footnote}}%
\makeatother
\usepackage{etoolbox} %


\newcommand*\linenomathpatch[1]{%
  \cspreto{#1}{\linenomath}%
  \cspreto{#1*}{\linenomath}%
  \csappto{end#1}{\endlinenomath}%
  \csappto{end#1*}{\endlinenomath}%
}

\linenomathpatch{equation}
\linenomathpatch{gather}
\linenomathpatch{multline}
\linenomathpatch{align}
\linenomathpatch{alignat}
\linenomathpatch{flalign}


\usepackage{amsfonts}
\usepackage{amssymb}
\usepackage{centernot}
\usepackage{graphicx}
\usepackage{algorithm}
\usepackage{algorithmic}
\usepackage[algo2e]{algorithm2e}
\newtheorem{thm}{Theorem}[section]
\newtheorem{prop}[thm]{Proposition}
\newtheorem{defn}[thm]{Definition}

\usepackage{fullpage}
\usepackage{lineno}

\usepackage{url}



\usepackage{color}
\newcommand{\amit}[1]{\textcolor{magenta}{#1}}

\title{Learning zeros of Fokker-Planck operators}


\begin{document}


\maketitle

 {\normalsize
 \centering
  \authorfootnotes
  Pinak Mandal\footnote{\thanks{Corresponding author: \texttt{pinak.mandal@icts.res.in}}}\textsuperscript{1}, Amit Apte\textsuperscript{1,2}

  \textsuperscript{1} International Centre for Theoretical Sciences - TIFR, Bangalore 560089 India \par
  \textsuperscript{2}Indian Institute of Science Education and Research, Pune 411008 India\par \bigskip}

\begin{abstract}
In this paper we devise a deep learning algorithm to find non-trivial zeros of Fokker-Planck operators when the drift is non-solenoidal. We demonstrate the efficacy of our algorithm for problem dimensions ranging from 2 to 10. Our method scales linearly with dimension in memory usage. We show that this method produces better approximations compared to Monte Carlo methods, for the same overall sample sizes, even in low dimensions. Unlike the Monte Carlo methods, our method gives a functional form of the solution. We also demonstrate that the associated loss function is strongly correlated with the distance from the true solution, thus providing a strong numerical justification for the algorithm. Moreover, this relation seems to be linear asymptotically for small values of the loss function.
\end{abstract}




\section{Introduction}\label{sec-intro}From the motion of a particle suspended in a fluid \cite{karatzas1991brownian},  enzyme kinetics \cite{allen2010introduction} to dynamics of a stock price \cite{karoui1997non}, \cite{delong2013backward} evolving systems in the real worlds are often modelled as systems of ordinary differential equations propagating under the influence of additive noise. These models known as stochastic differential equations (SDE) \cite{oksendal2003stochastic}, \cite{gardiner2009stochastic}, \cite{strauss2017hitch} are directly linked to Fokker-Planck equations (FPE) \cite{risken1996fokker} or Kolmogorov forward equations that describe the evolution of probability density of the state vector. In a prequel \cite{mandal2023learning} we developed a deep learning algorithm to compute non-trivial zeros of high-dimensional Fokker-Planck operators in a mesh-free manner. In this paper we will devise an algorithm to compute solutions of high-dimensional time-dependent FPEs in a mesh-free manner. We will begin by noting an algorithm similar to the one used in \cite{mandal2023learning} to solve stationary FPEs (SFPE) fails for time-dependent FPEs. A widely adopted strategy for solving high-dimensional PDEs is to appeal to Feynman-Kac type formulae \cite{del2004feynman}, \cite{jefferies2013evolution}, since they allow pointwise calculation of solutions without requiring a mesh thus mitigating at least one aspect of the curse of dimensionality. For example, Kakutani's solution of Dirichlet problem for the Laplace operator \cite{kakutani1944131}, \cite{kakutani1944143}, Muller's walk-on-spheres method for Dirichlet problems \cite{muller1956some} and an analogous method called walk-on-stars for Neumann problems \cite{sawhney2023walk}, multi-level Picard iteration method for solving semilinear heat equations \cite{hutzenthaler2021multilevel} are all based on Feynman-Kac type formulae. In recent times  deep learning methods have been combined with the Feynman-Kac formula to solve high-dimensional PDEs \cite{han2018solving}, \cite{blechschmidt2021three}. Even though FPEs are semilinear, parabolic PDEs whose solutions are probability densities and Deep-BSDE method proposed in \cite{han2018solving} deals with semilinear, parabolic PDEs, generic FPEs pose many challenges that make them unapproachable for deep-BSDE. Non-Lipschitzness of drift functions leading to blow-up of SDE trajectories \cite{chow2014almost}, \cite{li2011lack} and unboundedness of the divergence of drift functions causing FPEs to dissatisfy one of the requirements for the Feynman-Kac formula are foremost amongst these challenges. In this paper we apply the Feynman-Kac formula on an auxiliary equation and combine the solution with the zero of the Fokker-Planck operator obtained through the method described in \cite{mandal2023learning} to produce the normalized solution to the time-dependent Fokker-Planck equation.  We will apply our method for problem dimensions ranging from $2$ to $10$ to verify its effectiveness in high dimensions.


As we have noted, real world systems are often modelled as SDEs and we often observe such systems partially due to limited resources and are tasked with determining the distribution of the state vector at a certain time given all the observations up to that time. This is known as the filtering problem in the field of data assimilation and is useful for a variety of topics - global positioning system, target tracking, monitoring infectious diseases, to name a few \cite{sarkka2023bayesian}. FPEs are naturally connected to data assimilation when the underlying dynamics is stochastic. We will show how this method can be used to to calculate the one-step filtering density in the nonlinear filtering problem. To that end we will focus on systems with attractors since such systems are often used as important test cases in the field of data assimilation \cite{carrassi2022data}.




\section{Problem statement}
\label{sec-prob}\subsection{Time-dependent FPEs}
Consider the Itô SDE with coefficients $\mu\in C^1(\mathbb R^d;\mathbb R^d)$ and $\sigma\in C(\mathbb R^d; \mathbb R^{d\times l})$ that are independent of time,
\begin{equation}
\begin{aligned}
    dX_t &= \mu(X_t)\,dt + \sigma(X_t)\,dW_t\\
    X_0&\sim p_0\label{eq:SDE-0--dynamic-fp}
\end{aligned}
\end{equation}
such that $D=\frac{1}{2}\sigma\sigma^\top$ is a positive definite matrix for all inputs. Such an SDE is directly related to the Fokker-Planck equation describing the evolution of the probability density of $X_t$ \cite{risken1996fokker}, \cite{bogachev2022fokker}, \cite{bris2008existence},
\begin{equation}
\begin{aligned}
    &\frac{\partial p}{\partial t} = \mathcal Lp \stackrel{\rm def}{=}
  -\nabla \cdot(\mu p) + {\rm tr}(D\odot\nabla^2p)=0,\qquad\mathbf x\in\mathbb R^d,\; t\in(0, T]\\
  &p(0, \mathbf x) = p_0(\mathbf x),\qquad\mathbf x\in\mathbb R^d\\
  &\int_{\mathbb R^d}p(t, \mathbf x)\,d\mathbf x = 1,\qquad t\in[0, T] 
\label{eq:FPE-0--dynamic-fp}
\end{aligned}
\end{equation}
where $\odot$ is the Hadamard product and $\nabla^2$ denotes Hessian. The operator $\mathcal L$ is known as the Fokker-Planck operator (FPO). Our goal is to solve \eqref{eq:FPE-0--dynamic-fp} in a mesh-free way in high-dimensions. In our examples we stick to $\sigma \equiv cI_d$ for some $c>0$ which lets us abuse notation and use $\sigma$ and $D$ as scalar quantities, note that however our final method can be easily applied to problems where $\sigma$ is indeed a function of space or non-diagonal with minor modifications. With this simplification \eqref{eq:FPE-0--dynamic-fp} becomes,
\begin{align}
    \frac{\partial p}{\partial t}=
  -\nabla \cdot(\mu p) + D\Delta p=0\label{eq:FPE-1--dynamic-fp}
\end{align}
where $\Delta$ is the Laplacian operator. An extensive amount of work has been done to find numerical solutions to FPEs over the years, for a brief overview the reader can see section 4 of \cite{mandal2023learning}.  
\subsection{One step filtering problem}\label{ssec-1-filter-prob--dynamic-fp}
Suppose we partially observe $X_t$ at discrete times $t=g, 2g,3g,\cdots$ with observation gap $g$. These observations are given by, 
\begin{align}
    y_k  = Hx_k + \eta_k,\qquad k=1,2,\cdots\label{eq:filtering-obs--dynamic-fp}
\end{align}
where $x_k = X_{kg}$, $H:\mathbb R^d\to\mathbb R^q$ is a projection matrix and $\eta_k\sim\mathcal N(0_q, \sigma_o^2I_q)$ are iid Gaussian errors in observation. Given all the observations up to time $t=gk$, the filtering problem asks us to compute the distribution of the state vector at that time i.e. $p(x_k|y_{1:k})$. Here, as a simple application, we will calculate the one-step filtering density $p(x_1|y_1)$.


\section{Previous works}\label{sec-prev-work}An extensive amount of work has been done on the topic of numerically solving Fokker-Planck equations. A large amount of these works are based on traditional PDE solving techniques like finite difference \cite{berezin1987conservative}, \cite{whitney1970finite}, \cite{sepehrian2015numerical} and finite element \cite{naprstek2014finite}, \cite{masud2005application} methods. For a comparison of these traditional methods the reader can look at this comparative study \cite{pichler2013numerical} by Pitcher et al where the methods have been applied to 2 and 3 dimensional examples. 


In recent times efforts have been made to devise methods that are applicable in dimensions higher than 3. Tensor decomposition methods  \cite{Hackbusch2005HierarchicalKT}, \cite{kolda2009tensor} are an important toolkit while dealing with high-dimensional problems and they are proving to be useful in designing numerical solvers for PDEs \cite{ballani2013projection}, \cite{kressner2010krylov}.  For stationary Fokker-Planck equations Sun and Kumar proposed a tensor decomposition and Chebyshev spectral differentiation based method \cite{sun2014numerical} in 2014. In this method drift functions are approximated with a sum of functions that are separable in spatial variables, an well-established paradigm for solving PDEs. The differential operator for the stationary FPE is then discretized and finally a least sqaures problem is solved to find the final solution. The normalization is enforced via addition of a  penalty term in the optimization problem. The high-dimensional integral for the normalization constraint in this method is replaced with  products of one dimensional integrals and therefore becomes computable.   

In 2017 Chen and Majda proposed another hybrid method \cite{chen2018efficient} that utilizes both kernel and sample based density approximation to solve FPEs that originate from a specific type of SDE referred to as a conditional Gaussian model,
\begin{equation}
\begin{aligned}
    d\mathbf{u_I} = [A_0(t, \mathbf{u_I}) + A_1(t, \mathbf{u_I})\mathbf{u_{II}}]\,dt + \Sigma_I(t, \mathbf{u_I})\,dW_I(t)\\
    d\mathbf{u_{II}} = [a_0(t, \mathbf{u_I}) + a_1(t, \mathbf{u_I})\mathbf{u_{II}}]\,dt + \Sigma_{II}(t, \mathbf{u_I})\,dW_{II}(t)\label{eq:conditional-Gaussian-SDE}
\end{aligned}
\end{equation}
This special structure of the SDE allows one to approximate $p(\mathbf{u_{II}}(t))$ as a Gaussian mixture with parameters that satisfy auxiliary SDEs. $p(\mathbf{u_I}(t))$ is approximated with a non-parametric kernel based method. Finally the joint distribution $p(\mathbf{u_I}(t), \mathbf{u_{II}}(t))$ is computed with a hybrid expression. Using this method Chen and Majda computed the solution to a 6 dimensional conceptual model for turbulence. Note that, among our examples only L63 falls under this special  structure.

In recent years machine learning has also been applied to solve SFPEs. In 2019 Xu et al solved  2 and 3 dimensional stationary FPEs with deep learning \cite{xu2020solving}. Their method enforced normalization via a penalty term in the loss function that represented a Monte-Carlo estimate of the solution integrated over $\mathbb R^d$. Although simple and effective in lower dimensions, this normalization strategy loses effectiveness in higher dimensions. Zhai et al \cite{zhai2022deep} have proposed a combination of deep learning and Monte-Carlo method to solve stationary FPEs. The normalization constraint here is replaced with a regularizing term in the loss function which tries to make sure the final solution is close to a pre-computed Monte-Carlo solution. This strategy is more effective than having to approximate high-dimensional integrals and the authors successfully apply their method on Chen and Majda's 6 dimensional example.








\section{Overview of deep learning}
\label{sec-learning}
In this section we describe the general process of \textit{learning} a solution to a partial differential equation. The strategy described here will be an integral part of the final algorithm. In what follows next, we see how to see solve a generic PDE independent of time on a bounded domain with a Dirichlet boundary condition in a \textit{physics-informed} manner.  The interested reader can see \cite{raissi2019physics}, \cite{blechschmidt2021three}, \cite{sirignano2018dgm} for more discussions. In the next few subsections we keep simplifying our PDE problem until it finally becomes solvable on a computer.

\subsection{From PDE to optimization problem} In the context of machine learning, \textit{learning} refers to solving an optimization problem. So to solve our PDE with deep learning we first transform it into an optimization problem. Suppose our 2nd order PDE looks like, 
\begin{equation}
\begin{aligned}
    &\mathcal Lf(\mathbf x) = 0,\quad \mathbf x\in\Omega \,, \\
    &f(\mathbf x) = g(\mathbf x),\quad \mathbf x\in\partial\Omega \,,
\end{aligned}\label{eq:generic-pde}
\end{equation}
and just like before we are interested in finding a solution in $W^{1,2}(\Omega)\cap C^2(\Omega)$. Instead of trying to solve \eqref{eq:generic-pde} a popular strategy is to try to solve the following problem (see for example \cite{sirignano2018dgm}),
\begin{align}
    \min_{f\in W^{1,2}(\Omega)\cap C^2(\Omega)}\left[\int_\Omega (\mathcal L f)^2 + \int_{\partial\Omega}(f-g)^2\right]\label{eq:generic-pde-opt}
\end{align}
The choice of function space ensures one-to-one correspondence between the solutions of the PDE and the optimization problem.

\subsection{From infinite-dimensional search space to finite-dimensional search space}\label{ssec-infinite-to-finite} To solve a problem on a machine with finite resources we need to finitize the infinite aspects of the problem. We then solve the finitized problem which preferably approximates the original problem well to get an approximate solution to the orginal problem. For \eqref{eq:generic-pde-opt} our search space $W^{1,2}(\Omega)\cap  C^2(\Omega)$ is infinite dimensional which we need to replace with a finite dimensional search space.
In order to finitize the dimension of the search space we appeal to universal approximation theorems that say neural networks of even the simplest architectures are dense in continuous functions, see for example theorem 3.2 in \cite{kidger2020universal} or proposition 3.7 in \cite{pinkus1999approximation}. Universal approximation theorems typically allow networks to have either arbitrary depth or arbitrary width in order to achieve density \cite{pinkus1999approximation}, \cite{de2021approximation}. But the sets of neural networks with arbitrary depth or width are still infinite dimensional and therefore are infeasible to work with. In practice, we fix an architecture $\mathcal A$ with a fixed number of layers and trainable parameters and work with the following set instead.
\begin{align}
    S_{\mathcal A}\stackrel{\rm def}{=}\{n^{\mathcal A}_\theta: \theta\in \mathbb R^C\}\label{eq:search-space-net}
\end{align}
Here $n^{\mathcal A}_\theta$ is a network with architecture $\mathcal A$ with trainable parameters $\theta$ and $C$ is the total number of trainable parameters or the size of $\theta$. Since $C$ is fixed, $S_{\mathcal A}$ has a one-to-one correspondence with $\mathbb R^C$ and therefore is finite-dimensional. Even though we lose the density argument while working with $\theta$ of fixed size, in recent times it has been shown that sets like $S_\mathcal A$ are not closed in $W^{1, 2}(\Omega)$ and can, in principle, be used as good function approximators, see \cite{mahan2021nonclosedness} for a detailed discussion. In the following discussion we suppress the architecture and use $n^{\mathcal A}_\theta$ and $n_\theta$ interchangably for notational convenience. After restricting our search space to \eqref{eq:search-space-net},  \eqref{eq:generic-pde-opt} becomes,
\begin{align}
    \min_{\theta\in\mathbb R^C}\left[\int_{\Omega} (\mathcal L n_\theta)^2 + \int_{\partial\Omega}(n_\theta-g)^2\right]\label{eq:generic-pde-opt-theta}
\end{align}

\subsection{From integrals to sums} 
The domain $\Omega$ in our examples will often be of such a dimension that will make computing the integrals in \eqref{eq:generic-pde-opt-theta} extremely challenging. To deal with this we will replace the integrals in \eqref{eq:generic-pde-opt-theta} with Monte-Carlo sums as follows,

\begin{align}
    \min_{\theta\in\mathbb R^C}\left[\frac{1}{N}\sum_{j=1}^N (\mathcal L n_\theta(\mathbf x_j))^2 + \frac{1}{M}\sum_{j=1}^M(n_\theta(\mathbf y_j)-g(\mathbf y_j))^2\right]\label{eq:generic-opt-final}
\end{align}
where $\{\mathbf x_j\}_{j=1}^N$, $\{\mathbf y_j\}_{j=1}^M$ are uniform samples from $\Omega$ and $\partial \Omega$ respectively.

\subsection{Finding the optimal parameters}\label{ssec-finding-theta}
We simply perform gradient descent with respect to $\theta$ to find the optimal network for the problem \eqref{eq:generic-opt-final}. The Monte-Carlo sample sizes should be dictated by the hardware available to the practitioner.  In our experiments $N=M=1000$. In higher dimensions these choices are not enough to capture the original integrals entirely in one go. In that case \eqref{eq:generic-opt-final} can interpreted as trying to find a network that satisfies the original problem \eqref{eq:generic-pde} at the specified points $\{\mathbf x_j\}_{j=1}^N, \{\mathbf y_j\}_{j=1}^M$, which we can refer to as \textit{collocation points}. But in our experiments we try to the learn the solution on the entire domain as thoroughly as possible and so we resample the domain every few training iterations. So even though we are limited in sample-size by our hardware, we can shift the burden on space or memory to time or number of training iterations and adequately sample the entire domain. This principle of space-time trade-off is ubiquitous in machine learning \cite{buduma2022fundamentals} and comes in many different flavours like mini-batch gradient descent, stochastic gradient descent etc. Even though in this paradigm we are not training our network with typical input-output pairs, our method can be thought of as a variant of the mini-batch gradient descent.

\subsection{Why deep learning}
Deep learning in this context refers to learning an approximate solution to \eqref{eq:generic-pde} with the outlined method with an architecture $\mathcal A$ that is \textit{deep} or has many hidden layers. Deep networks are more efficient as approximators than shallow networks in the sense that they require far fewer number of trainable parameters to achieve the same level of approximation, for a discussion see section 5 of \cite{holstermann2023expressive}. Now that we have described the general procedure of \textit{deep learning} a solution to a PDE, we will pause briefly to point out some benefits and demerits of this approach. Deep learning has, like any other method some disadvantages. 
\begin{itemize}
    \item Deep learning is slower and less accurate for lower dimensional problems for which standard solvers exist and have been in consistent development for many decades. 
    \item Most modern GPUs are optimized for computation with single-precision or float32 numbers. This is efficient for rendering polygons or other image processing tasks which are the primary reasons GPUs were invented \cite{peddie2023history} but float32 might be not accurate enough for scientific computing. Although not ideal, this problem will most likely disappear in the future.
    \item The objective or \textit{loss} function used in a typical problem might not be convex and hence difficult to deal with \cite{krishnapriyan2021characterizing}, \cite{basir2022investigating}. 
\end{itemize}
But even with these disadvantages, the benefits of deep learning make it a worthwhile tool for solving PDEs.
\begin{itemize}
    \item Since we don't need to deal with meshes or grids in this method, we can mitigate the curse of dimensionality in memory. It will be clear from our experiments that the size of the network or $C$ does not need to grow exponentially with the dimensions. This method lets one compute the solution at collocation points but if one wants to compute the solution over the entire domain, one needs to sample the entire domain thoroughly which can be done in a sequential manner without requiring more memory as discussed in~\ref{ssec-finding-theta}.
    \item All derivatives are computed with automatic differentiation and therefore are accurate upto floating point errors. Moreover, finite difference schemes do not satisfy some fundamental properties of differentiation e.g. the product rule \cite{ranocha2019mimetic}. With automatic differentiation one does not have to deal with such problems. 
    \item If one computes the solution over the entire domain, the solution is obtained in a functional form which can be differentiated, integrated etc.
    \item Other than a method for sampling no modifications are required for accommodating different domains. 
\end{itemize}


\section{The algorithm}
\label{sec-algo}In this section we outline the algorithm for learning zeros of FPOs. But before that we go through the primary challenges and ways to mitigate them.

\subsection{Unboundedness of the problem domain}\label{ssec-unbounded-domain--steady-fp} We can try the same procedure as outlined in section~\ref{sec-learning--steady-fp} to find a non-trivial zero of $\mathcal L$. But computationally we can only deal with a bounded domain. Hence we focus on a compact domain which contains most of the mass of the solution to \eqref{eq:SFPE-0--steady-fp}. We refer to this domain as the \textit{domain of interest} $\Omega_I$ in the following discussion. \textcolor{magenta}{We note that the support of non-trivial zeros of $\mathcal L$ will usually be unbounded and of course we do not know the domain that may contain most of the mass. Thus the choice of $\Omega_I$ needs to be informed by some a priori knowledge of the solution, which in the examples we discuss is related to some attracting set of the deterministic system associated to the drift term $\mu$, i.e., the first term in~\eqref{eq-sde--steady-fp}. We do not need a precise knowledge of such an attracting set. But the smaller the domain $\Omega_I$, the more efficient will be the proposed method which requires uniform samples from $\Omega_I$.}
%{We note two points about the choice of $\Omega_I$: (i) The support of non-trivial zeros of $\mathcal L$ will usually be unbounded}


\subsection{Existence of the trivial solution}\label{ssec-exist-0--steady-fp} \textcolor{magenta}{Since $\mathcal L$ is a linear operator, zero is a trivial solution: $\mathcal L 0 = 0$. We also note that if $\nabla \cdot \mu \not\equiv 0$, then no other constant function (other than zero) is a zero of $\mathcal L$.} Since we want to find a non-trivial zero of $\mathcal L$, we would like avoid the learning the zero function during the training of the network. To deal with this problem \cite{zhai2022deep} added a regularization term that used approximate solutions of \eqref{eq:SFPE-0--steady-fp} found using Monte-Carlo. Here we propose a method that does not require a priori knowing an approximate solution. Consider the operator $\mathcal L_{\rm log}$ instead.
\begin{align}
    \mathcal L_{\rm log}f \stackrel{\rm def}{=} e^{-f}\mathcal L e^f\label{eq:def-log-FPO--steady-fp}
\end{align}
\textcolor{magenta}{Note that if $f$ is a zero of $\mathcal L_{\rm log}$, then $p = e^f$ is a zero of $\mathcal{L}$. Further, if $f$ is bounded below on some domain within $\Omega_I$, then $p$ is a non-trivial zero of $\mathcal{L}$.} Thus we can look for a zero of $\mathcal L_{\rm log}$ to find a non-trivial zero of $\mathcal L$. Recalling \eqref{eq:log-factor-V--steady-fp} we see that,
\begin{align}
    \mathcal L_{\rm log}f=
    -\nabla\cdot \mu - \mu \cdot \nabla f + D\left(\|\nabla f\|_2^2 + \Delta f\right)\label{eq:log-FPO--steady-fp}
\end{align}
We again note that when $\nabla\cdot\mu\not\equiv0$, then any constant function can not be a zero of $\mathcal L_{\rm log}$. 


OLDER VERSION (the iff below is not correct; also any (nonzero) constant is NOT a zero of L either so the first argument is not very strong  ...) ---- Recalling \eqref{eq:log-factor-V--steady-fp} we see that,
\begin{align}
    \mathcal L_{\rm log}f=
    -\nabla\cdot \mu - \mu \cdot \nabla f + D\left(\|\nabla f\|_2^2 + \Delta f\right)\label{eq:log-FPO-old--steady-fp}
\end{align}
Since $\nabla\cdot\mu\not\equiv0$, any constant function can not be a zero of $\mathcal L_{\rm log}$. $p$ is a zero of $\mathcal L$ iff either $p\equiv0$ or $\log p$ is a zero of $\mathcal L_{\rm log}$. So we can look for a zero of $\mathcal L_{\rm log}$ to find a non-trivial zero of $\mathcal L$. ---- END OLDER VERSION

\subsection{The steady state algorithm}
The procedure outlined in section~\ref{sec-learning--steady-fp} together with the modifications in sections~\ref{ssec-unbounded-domain--steady-fp}-\ref{ssec-exist-0--steady-fp} immediately yield the following loss function. \textcolor{magenta}{ADD the argument of LHS below (done)}
\begin{align}
    L_{\log}(\theta) = \frac{1}{N}\sum_{i=1}^N\mathcal L_{\rm log}(n_\theta(\mathbf x_i))^2\label{eq:def-steady-loss--steady-fp}
\end{align}
where $\{\mathbf x_i\}_{i=1}^N$ is a uniform sample from $\Omega_I$. Accordingly, the final procedure for finding a non-trivial zero of $\mathcal L$ is given in algorithm~\ref{algo:steady--steady-fp}.
%%%
\begin{algorithm}[t!]
Select the desired architecture for $n_\theta$.\\
Select resampling interval $\tau$.\\
Select an adaptive learning rate $\delta(k)$ and the number of training iterations $E$. 
Sample $\{\mathbf x_i\}_{i=1}^N$ from $\Omega_I$, the domain of interest.\\
\For {$k=1,2\cdots, E$}{
Compute $\nabla_\theta L_{\log}=\frac{1}{N}\sum_{i=1}^N\nabla_\theta(\mathcal{L}_{\log}(n_\theta(\mathbf x_i))^2)$\\
where $\mathcal L_{\log}f = -\nabla\cdot \mu - \mu \cdot \nabla f + D\left(\|\nabla f\|_2^2 + \Delta f\right)$\\
Update $\theta\leftarrow\theta - \delta(k) \nabla_{\theta}L_{\log}$\\
\If{$k\text{ is divisible by }\tau$}{Resample $\{\mathbf x_i\}_{i=1}^N$ from $\Omega_I$}
}
$e^{n_\theta(\mathbf x)}$ is a non-trivial zero of $\mathcal{L}$.\\
Optional: Approximate $Z\leftarrow\int_{\mathbb R^d}e^{n_\theta(\mathbf x)}\,d\mathbf x$.\\
$\frac{1}{Z}e^{n_\theta(\mathbf x)}$ is the learned, normalized steady state.
\caption{The steady state algorithm}
\label{algo:steady--steady-fp}
\end{algorithm}
%%%
In the following sections we describe in detail the network architecture and optimizer used in our experiments. Since the final solution is represented as $e^{n_\theta(\mathbf x)}$, we have automatically secured positivity of the solution.

\textcolor{magenta}{in general, is it not better to add another stopping criterion - how small is $L_{log}(\theta)$??? should we mention that somewhere???}

\subsection{Architecture}\label{ssec-architecture--steady-fp}
We choose the widely used LSTM \cite{sherstinsky2020fundamentals}, \cite{vennerod2021long} architecture described below for our experiments. This type of architecture rose to prominence in deep learning because of their ability to deal with the vanishing gradient problem, see section IV of \cite{sherstinsky2020fundamentals}, section 2.2 of \cite{vennerod2021long}. A variant of this architecture has also been used to solve PDEs \cite{sirignano2018dgm}. This kind of architectures have been shown to be universal approximators \cite{schafer2006recurrent}. We choose this architecture simply because of how \textit{expressive} they are. By  expressivity of an architecture we imply its ability to approximate a wide range of functions and experts have attempted to formalize this notion in different ways in recent times \cite{lu2017expressive}, \cite{raghu2016survey},  \cite{raghu2017expressive}. There are architectures that are probability densities by design i.e. the normalization constraint in \eqref{eq:SFPE-0--steady-fp} is automatically satisfied for them, see for example \cite{uria2013rnade}, \cite{papamakarios2019neural}. But our experiments suggest these architectures are not expressive enough to learn solutions to PDEs efficiently since the normalization constraint makes their structure too rigid. Other than the difficulty in implementing the normalization constraint numerically, this is another reason why we choose to focus on learning a non-trivial zero of $\mathcal L$ rather than solving \eqref{eq:SFPE-0--steady-fp}. LSTM networks on the other hand are expressive enough to solve all the problems listed in section~\ref{sec-examples--steady-fp}.

Below we define the our architecture in detail. Here $\mathbf 0_k$ implies a zero vector of dimension $k$ and $\odot$ implies the Hadamard product.
\begin{align}
    i\in\{1,2,&\cdots, L\}\\
    \mathtt c_0(\mathbf x)\stackrel{\rm def}{=}&\,\mathbf 0_m\label{eq:layer-c0--steady-fp}\\
    \mathtt h_0(\mathbf x)\stackrel{\rm def}{=}&\,\mathbf 0_d\label{eq:layer-h0--steady-fp}\\
    \mathtt f_i(\mathbf x) \stackrel{\rm def}{=}& \mathtt A(\mathtt W_f^{(i)}\mathbf x + \mathtt U_f^{(i)}h_{i-1}(\mathbf x) + \mathtt b_f^{(i)})\label{eq:layer-f--steady-fp}\\
    \mathtt g_i(\mathbf x) \stackrel{\rm def}{=}& \mathtt A(\mathtt W_g^{(i)}\mathbf x + \mathtt U_g^{(i)}h_{i-1}(\mathbf x) + \mathtt b_g^{(i)})\label{eq:layer-g--steady-fp}\\
    \mathtt r_i(\mathbf x) \stackrel{\rm def}{=}& \mathtt A(\mathtt W_r^{(i)}\mathbf x + \mathtt U_r^{(i)}h_{i-1}(\mathbf x) + \mathtt b_r^{(i)})\label{eq:layer-r--steady-fp}\\
    \mathtt s_i(\mathbf x) \stackrel{\rm def}{=}& \mathtt A(\mathtt W_s^{(i)}\mathbf x + \mathtt U_s^{(i)}h_{i-1}(\mathbf x) + \mathtt b_s^{(i)})\label{eq:layer-s--steady-fp}\\
    \mathtt c_i(\mathbf x) \stackrel{\rm def}{=}&  \mathtt f_i(\mathbf x)\odot \mathtt c_{i-1}(\mathbf x) + \mathtt g_i(\mathbf x)\odot s_i(\mathbf x)\label{eq:layer-c--steady-fp}\\
    \mathtt h_i(\mathbf x) \stackrel{\rm def}{=}& \mathtt r_i(\mathbf x)\odot \mathtt A(\mathtt c_i(\mathbf x))\label{eq:layer-h--steady-fp}\\
    \mathtt d_L(\mathbf x)\stackrel{\rm def}{=}&\mathtt W^\top\mathbf x + \mathtt b\label{eq:layer-final--steady-fp}\\
    n^{\rm LSTM}_\theta \stackrel{\rm def}{=}& \mathtt d_L\circ \mathtt h_L\label{eq:def-LSTM--steady-fp} 
\end{align}
Here $\{\mathtt f_i, \mathtt g_i, \mathtt r_i, \mathtt s_i, \mathtt c_i, \mathtt h_i: i=1,\cdots,L\}\cup\{\mathtt d_L\}$ are the hidden layers and
\begin{align}
    \theta=\{\mathtt W_f^{(i)}, \mathtt U_f^{(i)}, \mathtt b_f^{(i)}, \mathtt W_g^{(i)}, \mathtt U_g^{(i)}, \mathtt b_g^{(i)}, \mathtt W_r^{(i)}, \mathtt U_r^{(i)}, \mathtt b_r^{(i)}, \mathtt W_s^{(i)}, \mathtt U_s^{(i)}, \mathtt b_s^{(i)}:
i=1,\cdots,L\}\cup\{\mathtt W, \mathtt b\}\label{eq:theta-composition--steady-fp}
\end{align}is the set of the trainable parameters. The dimensions of these parameters are given below,
\begin{align}
   \mathtt W_f^{(i)}, 
   \mathtt W_g^{(i)},  \mathtt W_r^{(i)},  \mathtt W_s^{(i)} \in&\;
   \mathbb R^{m\times d}\\
   \mathtt U_f^{(i)},
\mathtt U_g^{(i)},
\mathtt U_r^{(i)},
\mathtt U_s^{(i)}\in&
   \begin{cases}\mathbb R^{m\times d},\quad\text{ if }i=1 \\
   \mathbb R^{m\times m},\quad\text{otherwise}
   \end{cases}\\
   \mathtt b_f^{(i)},\mathtt b_g^{(i)},\mathtt b_r^{(i)},\mathtt b_s^{(i)}\in&\;\mathbb R^m\\
   \mathtt W\in\mathbb R^{m}, \mathtt b\in&\;\mathbb R
\end{align}
which implies the size of the network or cardinality of $\theta$ is
\begin{align}
    C=4m[d(L+1)+m(L-1)]+5m+1\label{eq:network-size--steady-fp}
\end{align}
Note that \eqref{eq:network-size--steady-fp} implies the size of the network grows only linearly with dimension which is an important factor for mitigating the curse of dimensionality. We use elementwise $\tanh$ as our activation function,
\begin{align}
    \mathtt A=\tanh\label{eq:activation-choice--steady-fp}
\end{align}
We use $m=50$ and $L=3$ for our experiments which implies our network has $6L+1=19$ hidden layers. We use the very popular Xavier or Glorot initialization \cite{glorot2010understanding}, \cite{datta2020survey} to initialize $\theta$. With that, the description of our architecture is complete. 

\subsection{Optimization}\label{ssec-optimization--steady-fp}
In our experiments we use the ubiquitous Adam optimizer \cite{kingma2014adam} which is often used in the PDE solving literature \cite{han2018solving}, \cite{zhai2022deep}, \cite{sirignano2018dgm}. We use a piece-wise linear decaying learning rate. Below $k$ denotes the training iteration and $\delta(k)$ is the learning rate.
\begin{align}
    \delta(k)=\begin{cases}
        5\times10^{-3},\quad\text{if } k<1000
        \\
        1\times10^{-3},\quad\text{if }1000\le k<2000\\
        5\times10^{-4},\quad\text{if }2000\le k<10000\\
        1\times10^{-4},\quad\text{if }k\ge10000\\
        \end{cases}\label{eq:learning-rate--steady-fp}
\end{align}
We stop training after reaching a certain number of iterations $E$ which varies depending on the problem. In all our experiments we use $N=1000$ as the sample size and $\tau=10$ as the resampling interval for algorithm~\ref{algo:steady--steady-fp}.



\section{Results}\label{sec-steady-res}
In this section we describe the results in a manner that parallels the examples in section~\ref{sec-example--dynamic-fp} with additional problem-specific details. We refer to the solutions obtained via algorithm~\ref{algo:hybrid--dynamic-fp} as the \textit{learned} solutions in the figures below. We use the Monte-Carlo method as detailed in algorithm~\ref{algo:MC--dynamic-fp} to produce reference solutions in 2 and 3 dimensions. But in the absence of analytical solutions, we refrain from calculating the difference between the reference solution and the solution produced by algorithm~\ref{algo:hybrid--dynamic-fp} since Monte-Carlo solutions can be significantly erroneous, sometimes by more than an order of magnitude compared to its counterpart, even for simple problems, as evidenced by figure~7.2 in the prequel \cite{mandal2023learning}. When dealing with high-dimensional PDEs, one widely accepted paradigm is to construct solutions that are \textit{statistically} accurate or have the correct coarse-grained structures, especially for particle-based methods in geophysics \cite{bosler2013particle}, \cite{chen2018efficient} and financial modelling \cite{cui2015particle}. We present our results in the same spirit. For each system we use a bimodal initial condition as described in section~\ref{sec-example--dynamic-fp}. Figure~\ref{fig:L63-0--dynamic-fp} shows the initial condition for the noisy Lorenz-63 system defined in \eqref{eq:p0-L63--dynamic-fp}. For ease of visualization we have integrated out the last coordinate. The symmetry of the initial condition implies the other 2D marginals are identical. The other initial conditions described in section~\ref{sec-example--dynamic-fp} are either identical or qualitatively similar. In order to produce 2D marginal densities we use Gauss-Legendre quadrature the details of which can be found in appendix~9.3 of \cite{mandal2023learning}.
\begin{figure}[!ht]
    \centering
\includegraphics[scale=0.41]{dynamic-fp/plots/L63-0.png}
    \caption{Initial condition for the noisy Lorenz-63 system.}
    \label{fig:L63-0--dynamic-fp}
\end{figure}

\subsection{10D ring system} Figure~\ref{fig:10D-time--dynamic-fp} shows the computed solution for the 2$n$D ring system presented in section~\ref{ssec-2nD--dynamic-fp} for $n=5$, at time $t=0.1$. More specifically, the left panel of figure~\ref{fig:10D-time--dynamic-fp} depicts the probability density when all but the variables $x_4, x_5$ are set to $0$. For better visualization and comparison with a reference solution, the \textit{learned} solution has been normalized in a way such that,
\begin{align*}
   \int_\mathbb{R}\int_\mathbb{R} p(0.1, 0, 0, 0, 0, x_4, x_5, 0, 0, 0, 0)\, dx_4\,dx_5=1 
\end{align*}
Here we have used $M=10$, $N=10^5$ and $\mathcal D=[-2,2]^{10}$ for algorithm~\ref{algo:hybrid--dynamic-fp}.

There are no analytical solutions for this problem and classical methods are unsuitable due to the curse of dimensionality. Therefore, in order to compute a reference solution we use the careful design of the problem itself. The system described by \eqref{eq:potential-2nD--dynamic-fp} and the initial condition \eqref{eq:p0-2nD--dynamic-fp} is essentially $n$ identical, decoupled $2$D systems and we can approximately solve this $2$D system with Monte-Carlo method which is depicted in the right panel of figure~\ref{fig:10D-time--dynamic-fp}. We use $10^7$ trajectories to generate the Monte-Carlo solution. Note that, to compute the \textit{learned} solution in figure~\ref{fig:10D-time--dynamic-fp} we use a neural network approximation of $p_\infty$ obtained via the method described in \cite{mandal2023learning} rather than using the analytical version of $p_\infty$. This demonstrates the viability of algorithm~\ref{algo:hybrid--dynamic-fp} in conjunction with algorithm~6.1 of \cite{mandal2023learning} for high dimensional problems.

\begin{figure}[!ht]
    \centering
\includegraphics[scale=0.32]{dynamic-fp/plots/10D-time.png}
    \caption{Solutions for the 10D time-dependent system at time $t=0.1$. The learned solution has been normalized such that $\int_\mathbb{R}\int_\mathbb{R} p(0.1, 0, 0, 0, 0, x_4, x_5, 0, 0, 0, 0)\, dx_4\,dx_5=1$. The right panel depicts the Monte-Carlo solution for the $2$D Fokker-Planck equation corresponding to the variables $x_4,x_5$. The learned and Monte-Carlo solutions were computed using $10^5$ (pointwise) and $10^7$ trajectories respectively.}
    \label{fig:10D-time--dynamic-fp}
\end{figure}

\subsection{Noisy Lorenz-63 system}\label{ssec-res-L63--dynamic-fp}
Figure~\ref{fig:L63-time--dynamic-fp} shows the solutions for the noisy Lorenz system defined by \eqref{eq:mu-L63--dynamic-fp} at time $t=0.03$ with $\mathcal D=[-10, 10]\times[-15, 15]\times[7, 28], \Omega=[-30, 30]\times[-40, 40]\times[0, 70]$. As seen in figure~\ref{fig:xi--dynamic-fp}, $\xi(t,  \mathcal D, \Omega)\approx 1$. For easier visualization we present the 2D marginals $p(t, x, y), p(t, y, z), p(t, z, x)$. To compute $p_\infty$ in a functional form for this problem we use algorithm~6.1 in \cite{mandal2023learning}. We use $M=3$ and $N=200$ for the learned solution and $10^7$ trajectories to generate the corresponding Monte-Carlo solution. This shows that we can produce solutions with algorithm~\ref{algo:hybrid--dynamic-fp} that are comparable to Monte-Carlo method using 
several orders of magnitude fewer trajectories for each point. Both methods use $0.01$ as the step length for Euler-Maruyama.
 \begin{figure}[!ht]
    \centering
\includegraphics[scale=0.21]{dynamic-fp/plots/L63-time.png}
    \caption{Solutions for the noisy Lorenz 63 system at time t=0.03. The learned and Monte-Carlo solutions were computed using $200$ (pointwise) and $10^7$ trajectories respectively.}
    \label{fig:L63-time--dynamic-fp}
\end{figure}

\subsection{Noisy Thomas system}
Figure~\ref{fig:Thomas-time--dynamic-fp} shows the solutions for the noisy Thomas system defined by \eqref{eq:mu-Thomas--dynamic-fp} at time $t=1.0$ with $\mathcal D=[-8, 8]^3$ and $\Omega=[-10, 10]^3$. We only present $p(t, x, y)$, noting that the symmetry of the of problem renders demonstrations of the other 2D marginals redundant. We use algorithm~6.1 in \cite{mandal2023learning} for computing $p_\infty$. We use $M=10, N=50$ for the learned solution and the Monte-Carlo counterpart is computed using $10^7$ trajectories which reinstates our intuition that computing the Feynman-Kac expectation requires far fewer trajectories compared to Monte-Carlo for a similar level of accuracy.  Both methods use $0.1$ as the step length for Euler-Maruyama. In figure~\ref{fig:xi--dynamic-fp} we see that $\xi(t, \mathcal D, \Omega)\approx0.916$. So even after letting nearly $8.4\%$ of the h-SDE trajectories escape $\Omega$, we achieve a reasonable approximation.
\begin{figure}[!ht]
    \centering
\includegraphics[scale=0.32]{dynamic-fp/plots/Thomas-time.png}
    \caption{Solutions for the noisy Thomas system at time t=1. The learned and Monte-Carlo solutions were computed using $100$ (pointwise) and $10^7$ trajectories respectively.}
    \label{fig:Thomas-time--dynamic-fp}
\end{figure}

\subsection{One step filter}
Suppose in the filtering problem described in section~\ref{ssec-1-filter-prob--dynamic-fp} we only observe $x,z$ coordinates of the noisy Lorenz 63 system and our observation noise standard deviation is $\sigma_o=5$. The one-step filtering density can be written as 
\begin{align}
    p(x_1|y_1) \propto p(x_1|x_0)p(y_1|x_1)
\end{align}
For a derivation see chapter 6 of \cite{sarkka2023bayesian} or \cite{doucet2009tutorial}. $p(x_1| x_0)$ can be thought of as the solution to the corresponding Fokker-Planck equation at time $t=g=0.03$ (the observation gap) with the initial condition being equal to the density of $x_0$,. We can calculate the likelihood $p(y_1|x_1)$ using $\sigma_o$ which lets us estimate the 2D marginals of the one-step filtering density for this problem. We calculate the solution to the Fokker-Planck equation with algorithm~\ref{algo:hybrid--dynamic-fp} and Monte-Carlo. The final one-step filtering densities are shown in figure~\ref{fig:L63-filter--dynamic-fp}. Computing the filtering density with Monte-Carlo in this way is akin to using the bootstrap particle filter, a popular nonlinear filtering algorithm, see chapter 11 of \cite{sarkka2023bayesian} or \cite{doucet2009tutorial} for more discussion on particle filters. Therefore, we refer to the Monte-Carlo estimate for the filtering density as the particle filter estimate in figure~\ref{fig:L63-filter--dynamic-fp}. We use the same $M, N$ and the same number of trajectories for Monte-Carlo as we did in section~\ref{ssec-res-L63--dynamic-fp}. It is interesting to note that the solution in figure~\ref{fig:L63-time--dynamic-fp} is bimodal whereas in the filtering density in figure~\ref{fig:L63-filter--dynamic-fp} one of the mode collapses after we make an observation.

\begin{figure}[!ht]
    \centering
\includegraphics[scale=0.21]{dynamic-fp/plots/L63-filter.png}
    \caption{One step filtering density for the noisy Lorenz 63 system. The learned and particle filter solutions were computed using $200$ (pointwise) and $10^7$ trajectories respectively.}
    \label{fig:L63-filter--dynamic-fp}
\end{figure}


\section{Conclusions}
\label{sec-conclusions}
In this work we devise a deep learning algorithm  for finding non-trivial zeros of $\mathcal L$ when the corresponding drift is non-solenoidal. We can summarize our results as follows. 
\begin{enumerate}
    \item Our choice of architecture is capable of learning zeros of Fokker-Planck operators across many different problems while scaling only linearly with dimension.
    \item Time taken per training iteration grows near-linearly with problem dimension.
    \item Apart from being able to produce solutions in a functional form which Monte Carlo is incapable of, for the same overall sample-size algorithm~\ref{algo:steady--steady-fp} produces more accurate solutions compared to Monte Carlo.
    \item How quickly algorithm~\ref{algo:steady--steady-fp} converges to a zero depends as much on the dimension as it does on the nature of the problem or the structure of $\mu$.
    \item By minimizing the loss we also get closer to a true non-trivial zero of the Fokker-Planck operator which justifies algorithm~\ref{algo:steady--steady-fp}.
    \item The loss and the distance from a true zero of the Fokker-Planck operator, even though structurally completely different, are strongly correlated. Moreover, they can be asymptotically linearly related for small values of the loss function.
\end{enumerate}
In a sequel we will show how we can solve time-dependent FPEs by using the zeros learned by algorithm~\ref{algo:steady--steady-fp}. The landscape of the loss defined in \eqref{algo:steady--steady-fp} poses many interesting geometric questions. For example, in case the nullspace of $\mathcal L$ is 1-dimensional do all the minima of $L_{\log}$ lie on a connected manifold of dimension $1$
or are they disconnected from each other? Such questions provide possible avenues for future research.


\section{Appendix}
\label{sec-appendix}
\subsection{Uniqueness of solution for example problems}\label{ssec-unique} In this section we prove that the example problems used here can not have more than one weak solution in $W^{1,2}(\mathbb R^d)$.
 We employ the method of Lyapunov function as described in \cite{huang2015steady} to arrive at uniqueness. First we begin with the prerequisites for this approach.
 \subsubsection{Lyapunov functions and uniqueness}
\begin{defn}
    Let $U \in C(\mathcal U)$ be a non-negative function and denote $\rho_M = \sup_{\mathbf x\in \mathcal U} U(\mathbf x)$,
called the essential upper bound of $U$. $U$ is said to be a compact function in $U$ if
\begin{align}
i)\; U (\mathbf x) < \rho_M,\quad \mathbf x \in U
\end{align}
and
\begin{align}
ii) \lim_{\mathbf x\to\partial U} U (\mathbf x) = \rho_M 
\end{align}
\end{defn} This definition of a compact function appears as definition 2.2 in \cite{huang2015steady}.
\begin{prop}
An unbounded, non-negative function $U\in C(\mathbb R^d)$ is compact iff 
\begin{align}
    \lim_{\|\mathbf \mathbf x\|_2\to+\infty} U(\mathbf x) = +\infty
\end{align}
\end{prop}
This proposition appears as propposition 2.1 in \cite{huang2015steady}.

\begin{defn}
    Let $U$ be a compact function in $C^2(\mathcal U)$ with essential upper bound $\rho_M$. $U$ is called a Lyapunov function in $\mathcal U$ with respect to $\mathcal L^*$ is $\exists\,\rho_m\in(0, \rho_M)$ and a constant $\gamma>0$ such that
    \begin{align}
        \mathcal L^* U(\mathbf x)\le -\gamma,\qquad\forall \mathbf x\in \mathcal U\setminus\overline{\{\mathbf x\in\mathcal U: U(\mathbf x)<\rho_m\}}
    \end{align}
    where $\mathcal L^*$ is the adjoint Fokker-Planck operator given by
    \begin{align}
        \mathcal L^*f = \mu\cdot\nabla f+ D\odot\nabla^2 f \label{eq:def-adjoint-FP-op}
    \end{align}
\end{defn}
This definition appears as definition 2.4 in \cite{huang2015steady}.
Now we are ready to state the main theorem that will help us prove uniqueness for our example problems.
\begin{thm}
    If the components of $\mu$ are in $L^2_{\rm loc}(\mathcal U)$ and there exists a Lyapunov function with respect to $\mathcal L^*$ in $C^2(\mathcal U)$ then \eqref{eq:SFPE-0} has a positive weak solution in the space $W^{1, 2}_{\rm loc}(\mathcal U)$. If, in addition,
the Lyapunov function is unbounded, the solution is unique in $\mathcal U$.
\end{thm}
This theorem appears as theorem $A$ in \cite{huang2015steady}. Since the components of $\mu$ are locally integrable for our example problems, all we need to do is find an unbounded Lyapunov function $U$ for proving existence and uniqueness in $W^{1,2}_{\rm loc}(\mathbb R^d)$ which in turn implies if a strong solution exists in $W^{1, 2}(\mathbb R^d)\cap C^2(\mathbb R^d)$, it must be unique.
\subsubsection{Uniqueness of solution for 2D ring system}\label{sssec-2D-unique}
Setting 
\begin{align}
\mathcal U &= \mathbb R^2\\
U(x, y) &= x^2+y^2\\
\rho_m &= \frac{1}{2}+\sqrt{D+1}\\
\gamma &= 4D+6\\
\end{align}
we see that,
\begin{align}
    \mathcal L^*U +\gamma = -8\left(x^2+y^2-\frac{1}{2}\right)^2 + 8(D+1)
\end{align}
and,
\begin{align}
    \mathcal U\setminus\overline{\{\mathbf x\in\mathcal U: U(\mathbf x)<\rho_m\}} = \left\{(x,y)\in\mathbb R: x^2+y^2>\rho_m\right\}
\end{align}
In  $\left\{(x,y)\in\mathbb R: x^2+y^2>\rho_m\right\}$, 
\begin{align}
    \mathcal L^*U+\gamma \le 0
\end{align}
and therefore $U$ is an unbounded Lyapunov function for the 2D ring system which guarantees uniqueness of solution \eqref{eq:grad-sol}.

\subsubsection{Uniqueness of solution for L63 system}\label{sssec-L63-unique}
Setting,
\begin{align}
U(x, y, z) = \rho x^2 +\alpha y^2 + \alpha(z-2\rho)^2
\end{align}
we see that
\begin{align}
    \mathcal L^*U &= -2\alpha\rho x^2 - 2\alpha y^2 -2\alpha\beta z^2 + 4\alpha\beta\rho z + 2D(2\alpha+\rho)\\
    &=-2\alpha\rho x^2 - 2\alpha y^2 -\alpha\beta z^2 -\alpha\beta(z-2\rho)^2 + 4\alpha\beta\rho^2 + 2D(2\alpha+\rho)\\
    &\le -\rho x^2 -\alpha y^2 -\alpha(z-2\rho)^2 + 4\alpha\beta\rho^2 + 2D(2\alpha+\rho)\label{eq:L63-params-bigger-than-1}\\
    &= -U(x, y, z)+ 4\alpha\beta\rho^2 + 2D(2\alpha+\rho)
\end{align}
\eqref{eq:L63-params-bigger-than-1} is a consequence of $\alpha, \beta, \rho>1$. Now setting,
\begin{align}
    \gamma &= 1,\\
    \rho_m &= 4\alpha\beta\rho^2 + 2D(2\alpha+\rho)+1
\end{align} we see that in $\{U>\rho_m\}$,
\begin{align}
    \mathcal L^*U +\gamma \le 0
\end{align}
So $U$ is an unbounded Lyapunov function for this system and we have a unique solution.
\subsubsection{Uniqueness of solution for Thomas system}\label{sssec-Thomas-unique}
Setting,
\begin{align}
    U(x, y, z) = x^2+y^2+z^2
\end{align}
we see that
\begin{align}
    \mathcal L^* U &= x\sin y + y\sin z + z\sin x - b(x^2+y^2+z^2) + 6D\\
    &\le \sqrt{3U}-bU + 6D\label{eq:CS-on-Thomas-unique}\\
    &= -b\left(\sqrt{U}-\frac{\sqrt{3}}{2b}\right)^2 +\frac{3}{4b}+6D
\end{align}
\eqref{eq:CS-on-Thomas-unique} follows from Cauchy Schwarz inequality. Setting,
\begin{align}
    \gamma &= \frac{1}{4b},\\
    \rho_m &= \left(\frac{\sqrt{3}}{2b}+\frac{\sqrt{1+6bD}}{b}\right)^2
\end{align}
we see that in $\{U>\rho_m\}$,
\begin{align}
    \mathcal L^*U +\gamma \le 0
\end{align}
So $U$ is an unbounded Lyapunov function for this system and we have a unique solution.

\subsection{Monte Carlo algorithm}\label{ssec-MC-algo}
The time-dependent FPE given by 


\begin{equation}
\begin{aligned}
    &\frac{\partial  p(t, \mathbf x)}{\partial t} =\mathcal L p(t, \mathbf x),\qquad\mathbf x\in\mathbb R^d,\; t\ge0\\&p(0, \mathbf x)=p_0(\mathbf x),\qquad\mathbf x\in\mathbb R^d\\
    &\int_{\mathbb R^d}p(t,\mathbf x)\,d\mathbf x = 1,\qquad\forall\;t\ge0
    \label{eq:FPE-0}
\end{aligned}
\end{equation}


gives us the probability density of the random process $X_t$ which is governed by the SDE,
\begin{equation}
\begin{aligned}
    &dX_t=\mu\,dt+\sigma\,dW_t\\
    &X_0\sim p_0\label{eq:SDE-0}
\end{aligned}
\end{equation}where $\{W_t\}$ is the standard Wiener process, see for example chapters 4, 5 of \cite{gardiner2009stochastic}. We can evolve $\eqref{algo:steady}$ up to sufficiently long time using Euler-Maruyama method \cite{kloeden1992stochastic} to approximate the steady state solution of \eqref{eq:FPE-0} or the solution of \eqref{eq:SFPE-0} as follows. Here $\mathcal N$ denotes the multivariate normal distribution.
\begin{algorithm}[!ht]
Sample $\{ X_0^{(i)}\}_{i=1}^N\sim p_0$.\\
Set the time-step $h$.\\
Set the number of steps $S$.\\
\For {$k=1,2\cdots, S$}{
 Sample $w^i_k\sim\mathcal N(\mathbf 0_d, h I_d)\;\;\forall\;i$\\
 $ X_k^{(i)}\leftarrow  X_{k-1}^{(i)} + \mu\left(X_{k-1}^{(i)}\right)h + \sigma w_k^i\;\;\forall\;i$\\
}
Subdivide the domain of interest $\Omega_I$ into $d$-dimensional boxes.\\ Count the number of $X^{(i)}_{S}$ that are in a box to estimate the stationary density at the center of the box.
\caption{Monte Carlo algorithm}\label{algo:MC}
\end{algorithm}
Note that in case of a unique solution of \eqref{eq:SFPE-0}, many choices of $p_0$ can lead to the stationary solution. In all our examples, it suffices to choose $p_0$ to be the standard $d$-dimensional normal distribution.

\subsection{Integration error for $n$-point Gauss-Legendre rule}\label{ssec-error-GL}
Suppose we are trying to integrate a smooth function $f(x)$ over $\left[a-\frac{h}{2}, a+\frac{h}{2}\right]$ with $n$-point Gauss-Legendre rule where $h\in(0, 1]$. Let us denote $I[f]$ to be the Gauss-Legendre approximation of $\int_{a-\frac{h}{2}}^{a+\frac{h}{2}} f(x)\,dx$. Recalling that $n$-point Gauss-Legendre gives us exact integrals for polynomial of degree $\le 2n-1$ and using the Lagrange form of Taylor remainder we see that, 
\begin{align}
    \left|I[f]-\int_{a-\frac{h}{2}}^{a+\frac{h}{2}} f(x)\,dx\right| &\le MI\left[\frac{(x-a)^{2n}}{(2n)!}\right]+M\int_{a-\frac{h}{2}}^{a+\frac{h}{2}} \frac{(x-a)^{2n}}{(2n)!}\,dx\label{eq:diff-GL-true}
\end{align}
where $|f^{(2n)}(x)|\le M\;\;\forall\;\;x\in\left[a-\frac{h}{2}, a+\frac{h}{2}\right]$. To bound the first term on the RHS of \eqref{eq:diff-GL-true} we can use the fact that if 
\begin{align}
    I[f] = \sum_{i=1}^nw_if(x_i)
\end{align}
then,
\begin{align}
    &I[1] = \int_{a-\frac{h}{2}}^{a+\frac{h}{2}} 1 \,dx = h\\
    \implies&\sum_{i=1}^nw_i = h\le1\\
    \implies&I\left[\frac{(x-a)^{2n}}{(2n)!}\right]\le\frac{1}{(2n)!}\left(\frac{h}{2}\right)^{2n}
\end{align}
Therefore,
\begin{align}
    \left|I[f]-\int_{a-\frac{h}{2}}^{a+\frac{h}{2}} f(x)\,dx\right|\le \frac{M}{(2n)!}\left(\frac{h}{2}\right)^{2n}+\frac{2M}{(2n+1)!}\left(\frac{h}{2}\right)^{2n+1}\le\frac{2M}{(2n)!}\left(\frac{h}{2}\right)^{2n}
\end{align}
 


\bibliographystyle{siamplain}
\bibliography{ref}
\end{document}