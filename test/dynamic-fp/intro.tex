From the motion of a particle suspended in a fluid \cite{karatzas1991brownian},  enzyme kinetics \cite{allen2010introduction} to dynamics of a stock price \cite{karoui1997non}, \cite{delong2013backward} evolving systems in the real worlds are often modelled as systems of ordinary differential equations propagating under the influence of additive noise. These models known as stochastic differential equations (SDE) \cite{oksendal2003stochastic}, \cite{gardiner2009stochastic}, \cite{strauss2017hitch} are directly linked to Fokker-Planck equations (FPE) \cite{risken1996fokker} or Kolmogorov forward equations that describe the evolution of probability density of the state vector. In a prequel \cite{mandal2023learning} we developed a deep learning algorithm to compute non-trivial zeros of high-dimensional Fokker-Planck operators in a mesh-free manner. In this paper we will devise an algorithm to compute solutions of high-dimensional time-dependent FPEs in a mesh-free manner. We will begin by noting an algorithm similar to the one used in \cite{mandal2023learning} to solve stationary FPEs (SFPE) fails for time-dependent FPEs. A widely adopted strategy for solving high-dimensional PDEs is to appeal to Feynman-Kac type formulae \cite{del2004feynman}, \cite{jefferies2013evolution}, since they allow pointwise calculation of solutions without requiring a mesh thus mitigating at least one aspect of the curse of dimensionality. For example, Kakutani's solution of Dirichlet problem for the Laplace operator \cite{kakutani1944131}, \cite{kakutani1944143}, Muller's walk-on-spheres method for Dirichlet problems \cite{muller1956some} and an analogous method called walk-on-stars for Neumann problems \cite{sawhney2023walk}, multi-level Picard iteration method for solving semilinear heat equations \cite{hutzenthaler2021multilevel} are all based on Feynman-Kac type formulae. In recent times  deep learning methods have been combined with the Feynman-Kac formula to solve high-dimensional PDEs \cite{han2018solving}, \cite{blechschmidt2021three}. Even though FPEs are semilinear, parabolic PDEs whose solutions are probability densities and Deep-BSDE method proposed in \cite{han2018solving} deals with semilinear, parabolic PDEs, generic FPEs pose many challenges that make them unapproachable for deep-BSDE. Non-Lipschitzness of drift functions leading to blow-up of SDE trajectories \cite{chow2014almost}, \cite{li2011lack} and unboundedness of the divergence of drift functions causing FPEs to dissatisfy one of the requirements for the Feynman-Kac formula are foremost amongst these challenges. In this paper we apply the Feynman-Kac formula on an auxiliary equation and combine the solution with the zero of the Fokker-Planck operator obtained through the method described in \cite{mandal2023learning} to produce the normalized solution to the time-dependent Fokker-Planck equation.  We will apply our method for problem dimensions ranging from $2$ to $10$ to verify its effectiveness in high dimensions.


As we have noted, real world systems are often modelled as SDEs and we often observe such systems partially due to limited resources and are tasked with determining the distribution of the state vector at a certain time given all the observations up to that time. This is known as the filtering problem in the field of data assimilation and is useful for a variety of topics - global positioning system, target tracking, monitoring infectious diseases, to name a few \cite{sarkka2023bayesian}. FPEs are naturally connected to data assimilation when the underlying dynamics is stochastic. We will show how this method can be used to to calculate the one-step filtering density in the nonlinear filtering problem. To that end we will focus on systems with attractors since such systems are often used as important test cases in the field of data assimilation \cite{carrassi2022data}.


