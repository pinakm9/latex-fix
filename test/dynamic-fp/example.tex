All the examples here parallel the examples that appear in the prequel \cite{mandal2023learning} where we call a system a \textit{gradient system} if the corresponding $\mu$ can be written as the gradient of some potential function,
\begin{align}
    \mu=-\nabla V\label{eq:grad-mu--dynamic-fp}
\end{align}
and otherwise we call it \textit{non-gradient system}. Gradient systems provided important test cases in the prequel \cite{mandal2023learning} while solving stationary FPEs since their steady states are known in analytical form. But for time-dependent FPEs analytical solutions are not known in general even if $\mu$ satisfies \eqref{eq:grad-mu--dynamic-fp}. 

\subsection{Gradient systems} We use the following gradient systems to verify the effectiveness of our algorithm in high-dimensions. A corresponding non-trivial zero of the Fokker-Planck operator was calculated in \cite{mandal2023learning} for each case.
\subsubsection{2D ring system}\label{ssec-2D--dynamic-fp}
For $V=(x^2+y^2-1)^2$ and $\mu=-\nabla V$ we get the following FPE,
\begin{align}
     \frac{\partial p}{\partial t}=4(x^2+y^2-1)\left(x\frac{\partial p}{\partial x}+y\frac{\partial p}{\partial y}\right) + 8(2x^2+2y^2-1)p + D\Delta p\label{eq:ring2D--dynamic-fp}
\end{align}
The corresponding ODE system has the unit circle as a global attractor. We solve this system for $D=1$. As the initial condition we use an equal Gaussian mixture with the components having centers at $\left(-\frac{1}{2},-\frac{1}{2} \right)$ and $\left(\frac{1}{2},\frac{1}{2} \right)$ with covariance matrix $\frac{1}{4}I_2$.
\begin{align}
    p_0(x, y) = \frac{1}{\pi}\exp\left(-2\left(x+\frac{1}{2}\right)^2-2\left(y+\frac{1}{2}\right)^2\right)+\frac{1}{\pi}\exp\left(-2\left(x-\frac{1}{2}\right)^2-2\left(y-\frac{1}{2}\right)^2\right)\label{eq:p0-2D--dynamic-fp}
\end{align}
This system has a unique solution, for a proof see appendix~\ref{ssec-unique--dynamic-fp}.
\subsubsection{2nD ring system}\label{ssec-2nD--dynamic-fp} We can daisy-chain the previous system to build decoupled systems in higher dimensions. In this case the potential is given by
\begin{align}
    V(\mathbf x)=\sum_{i=0}^{\frac{d}{2}-1}(x_{2i}^2+x_{2i+1}^2-1)^2,\qquad d=2n\label{eq:potential-2nD--dynamic-fp}
\end{align}
We use the following initial condition which can be obtained by daisy-chaining the initial condition in \eqref{eq:p0-2D--dynamic-fp}.
\begin{align}
    p_0(\mathbf x) = \prod_{i=0}^{\frac{d}{2}-1
    }\left[\frac{1}{\pi}\exp\left(-2\left(x_{2i}+\frac{1}{2}\right)^2-2\left(x_{2i+1}+\frac{1}{2}\right)^2\right)+\frac{1}{\pi}\exp\left(-2\left(x_{2i}-\frac{1}{2}\right)^2-2\left(x_{2i+1}-\frac{1}{2}\right)^2\right)\right]\label{eq:p0-2nD--dynamic-fp}
\end{align}
Since our algorithm does not differentiate between coupled and decoupled systems, this example serves as a great high-dimensional test case.  Although the analytical solution for this problem is not known, the decoupled nature of this problem implies that we can compare our solution to the 2nD ring system with the solution to the 2D ring system which can be easily computed with other methods (e.g. Monte Carlo) that are not efficient in higher dimensions. Here we solve this system for $n=1,2,3,4,5$ and $D=1$. This system has a unique solution which follows directly from the fact that the 2D ring system has a unique solution.
\subsection{Non-gradient Systems}
Even though the analytic solution for the gradient case is not known for time-dependent FPEs, the gradient structure of the drift makes solving time-dependent FPEs much \textit{easier} as we will see in section~\ref{ssec-limit--dynamic-fp}. Non-gradient systems on the other hand pose a much harder challenge due to the blow-up of auxiliary SDEs, see section~\ref{ssec-limit--dynamic-fp} for more details. Here we deal with the following non-gradient systems. A corresponding non-trivial zero of the Fokker-Planck operator was calculated in \cite{mandal2023learning} for each case.
\subsubsection{Noisy Lorenz-63 system}\label{ssec-63--dynamic-fp} The Lorenz-63 system was first proposed by Edward Lorenz 
as an oversimplified model for atmospheric convection \cite{lorenz1963deterministic}. The corresponding ODE possesses the famous butterfly attractor.
This system and its variants like Lorenz-96 are staple test problems in the field of data assimilation \cite{carrassi2022data}, \cite{yeong2020particle} which is why we also use this system for the calculation of one-step filtering density. We use the standard parameters to define the drift.
\begin{equation}
\begin{aligned}
    &\mu=[\alpha (y-x),\, x(\rho-z) - y,\, xy - \beta z]^\top\\
    &\alpha = 10 \,, \, \beta = \frac{8}{3}\,, \, \rho=28 \
\end{aligned}\label{eq:mu-L63--dynamic-fp}
\end{equation}
We solve this system for $D=50$ with the following Gaussian mixture as the initial condition,
\begin{align}
    p_0(\mathbf x)=\frac{1}{2\sqrt{(2\pi)^3}}\exp\left(-\frac{\|\mathbf x + 2\times\mathbf 1_3\|_2^2}{2}\right) + \frac{1}{2\sqrt{(2\pi)^3}}\exp\left(-\frac{\|\mathbf x - 2\times\mathbf 1_3\|_2^2}{2}\right)\label{eq:p0-L63--dynamic-fp}
\end{align}
where $\mathbf 1_3$ denotes the vector with all entries $1$ with respect to the standard basis in $\mathbb R^3$. This system has a unique solution, for a proof see appendix~\ref{ssec-unique--dynamic-fp}.
\subsubsection{Noisy Thomas system}\label{ssec-Thomas--dynamic-fp}
The deterministic or ODE version of this system was proposed by René Thomas~\cite{thomas1999deterministic}. This is a 3-dimensional system with cyclical symmetry in $x, y, z$ and the corresponding ODE system has a cyclically symmetric strange attractor.
\begin{equation}
\begin{aligned}
    &\mu=[\sin y - bx,\, \sin z - by,\, \sin x - by ]^\top\\
    &b =  0.2 
\end{aligned}\label{eq:mu-Thomas--dynamic-fp}
\end{equation}
We solve this system for $D=1$ with the following Gaussian mixture as the initial condition,
\begin{align}
    p_0(\mathbf x)=\frac{1}{2\sqrt{(2\pi)^3}}\exp\left(-\frac{\|\mathbf x + 2\times\mathbf 1_3\|_2^2}{2}\right) + \frac{1}{2\sqrt{(2\pi)^3}}\exp\left(-\frac{\|\mathbf x - 2\times\mathbf 1_3\|_2^2}{2}\right)\label{eq:p0-Thomas--dynamic-fp}
\end{align}
This system has a unique solution, for a proof see appendix~\ref{ssec-unique--dynamic-fp}.

To verify our method we will compare the solutions obtained by our method with Monte Carlo simulations of \ref{eq:SDE-0--dynamic-fp}, for details of the Monte Carlo algorithm see appendix~\ref{ssec-algo-MC--dynamic-fp}. 
